\chapter{Conclusions}
In this paper we found indications that users of social media are, to varying degrees, aware of the trade-off of personal information for access to the service, in this case Facebook. Despite the sacrifice of privacy, some people choose to use the service anyway, as the convenience and psychological satisfaction of using it outweighs the perceived cost. The cost of big data gathering is perceived as fairly low, due to users feeling anonymous in the masses of users, and due to trust in the justice system and data security. The sacrifice of privacy is, to some extent, accepted as a given in today’s digitalized society.

This attitude towards big data collection allows companies to make use of it and capitalize upon it, as long as users do not feel threatened or compromised. However, it still raises some ethical concerns in regards to the boundaries between users and companies’ access to their private lives.