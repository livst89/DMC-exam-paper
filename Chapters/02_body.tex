\chapter{Body}
This section of the report deals with the analysis of related work, relevant theory and methods used to investigate the research question. The final part of the section is a discussion of the findings.

\section{Related Work}
In the paper titled “Privacy in the Age of Big Data - A Time for Big Decisions” by Tene, O. \& Polonetsky, J. (2012)\todo{Ref}, they touch upon both the benefits and drawbacks of using big data. While it can be used to improve services and provide knowledge about large networks of people, the major concern is the risk of the data being “personally identifiable information”, leading to a breach in privacy (Tene \& Polonetsky, 2012)\todo{Ref}. The decision whether to allow the gathering and use of big data is boiled down to a value choice: Does the benefits outweigh the risk? Their conclusion is that emphasis must be put on balancing the benefit of data for businesses and researchers against individual privacy rights (Tene \& Polonetsky, 2012)\todo{Ref}.
In a survey study regarding social media and the impact of privacy concerns by \cite{imagined}, their findings show that an individual’s privacy concern is only a weak predictor of membership to a social network, such as Facebook. Privacy concerned individuals also participate in sharing a lot of personal information through the social media. Individuals with privacy concerns tend to trust their own ability to control access to the content they share, but the study also found that members have misconceptions about the size of the online community and the visibility of their own profiles.

\section{Theoretical Framework}
In the article “Critical Questions for Big Data: Provocations for a cultural, technological, and scholarly phenomenon” by danah boyd and Kate Crawford (2012)\todo{Ref}, they present six provocations about big data (boyd \& Crawford, 2012)\todo{Ref}. This paper will focus on their thoughts about big data as being accessible, but not necessarily ethical. By using their thoughts, we are able to explore and discuss the ethical element of Facebook collecting user data, without their users necessarily approving of it.

On the other hand, one might say that it is the responsibility of the user, as they are accepting the terms and conditions when joining the platform. It is interesting to relate the issue to the theoretician Anthony Giddens (1994)\todo{Ref}, who takes a more societal approach. He states that there are certain characteristics in the modern society wherein the individual exist. Giddens (1994 \& 1996)\todo{Ref} deals with how the modern society and the individuals in it influence one another. One of the characteristics Giddens (1994)\todo{Ref} defines is the disembedding of social systems. One might argue that the Facebook community is a disembedding of one's social network and social interaction, and further that Facebook can be considered an expert system within the area of data storage. According to Giddens (1994)\todo{Ref}, all disembedding relies on trust. We need to trust that Facebook does not abuse our personal information, and that they secure the personal information against hackers. This means that there are norms of trust in abstract systems in our modern society (Giddens, 1994)\todo{Ref}. The thoughts of Giddens (1994)\todo{Ref} can help us explore and understand the context of the individual and enlighten a perspective on the user’s trust in Facebook as an abstract system.

Furthermore, we want to understand what motivates Facebook users to accept the lack of online privacy. In order to do so, we will use Abraham Maslow’s (1943)\todo{Ref} theory of the hierarchy of needs. Maslow (1943)\todo{Ref} argues that any motivated behavior must be understood through basic needs, which he classifies by starting with fundamental physiological needs, such as sleep, food and sex. After comes the needs of safety, love, esteem and lastly self-articulation. When one need is satisfied, a new one will appear (Maslow, 1943)\todo{Ref}. We classify the need for Facebook as esteem and self-articulation. In modern society, a common reinterpretation of the model argues for the “opposite hierarchy”, meaning that within our society we are partly likely to take the fundamental needs for granted, because it is not something that we have to fight for to get. Therefore, the need for esteem and self-articulation through social media is getting more important, as the more fundamental needs are met.

\section{Methods}
The empirical foundation of this paper was a semi-structured focus group interview with five informants in the age between 20-35 years. An interview guide was prepared ahead of time to ensure topic coverage. The argument for choosing this method was to save time, since it is possible to produce a big amount of knowledge in a relatively short time. Also, the group dynamic of this method increases the memory of the informants, as their point of view can trigger new questions to investigate (Harboe, 2010)\todo{Ref}. This can either be done by the other informants or by the researches if they wish to explore the new aspect further. A challenge of the method is that informants can influence each other, resulting in withdrawn or distorted information (Harboe, 2010)\todo{Ref}.

The purpose of this method has been to enlighten the extent of awareness regarding one's behavior on Facebook and big data collection, and how important it is to the informants.

We are aware that we have conducted one focus group interview only. According to Bryman (2012)\todo{Ref}, the optimal amount would be between ten to fifteen focus group interviews, in order to achieve a better validity and accurate exploration of the research question. Therefore, we are not able to generalize our knowledge based on our investigation, but it has given us some interesting qualitative knowledge.

\subsection{Empirical Results}
The focus group had varying levels of concern regarding privacy. This ranged from not worrying too much about it to consciously being suspicious of it. They were all aware of Facebook owning uploaded content, but some were unsure if Facebook could sell photos to third party companies. None could remember the specifics of the terms of service when they signed up. Most were aware of the different terms for installing the Messenger app, but they used it anyway. Informant 1 said: “All that is free on the Internet, you are the product”. But they did not seem to care about it because they expressed a sense of distance to the intangible media. Informant 3 said: “Like, they are hungry and poor in Africa, but you turn a blind eye to it. I think it is the same way with Facebook”.

After showing a video clip about privacy on the Messenger app (DR, 2015)\todo{Ref}, they expressed concern about the apps access to their text messages and editing rights. Despite the new information, they would not change their use or behaviour. They choose to use the platform despite disliking the terms, due to wanting to be part of the social network. Informant 2 stated that: “The enjoyment of Facebook compensate for the bad stuff”.

When asked whether their social networks were aware of the privacy issues, they said no. Their overall impression was a general lack of privacy awareness. Even if people are concerned they do not act upon it. Informant 3 said:

“This is the world today. We just need to have trust in that the system works. Likewise with bank transfers and so on when we buy stuff online. This is just the way things are with the IT development and stuff. There is nothing else to do other than trust it, and trust that if something goes all wrong, there will be someone handling it”.

The informants also expressed a lack of action due to feeling powerless to change status quo. When considering alternatives to the existing form of Facebook, all respondents expressed interest in a paid social network service with better data privacy terms.

\section{Discussion}
As our results show the same indication as the study from 2006, even despite the fact that further incursions has been made to data privacy through the introduction of the Messenger app, it is interesting to consider the possible explanations for this behaviour. Relating the issue to Giddens\todo{Ref} provides a societal perspective, in that people have trust in the system to not misuse their data against them. It also indicates that the balance of Marslow’s\todo{Ref} hierarchy of needs has changed. As people’s most basic needs are satisfied, the needs which were previously of lower priority now carry more weight in people’s lives. By combining these two perspectives, has the individual’s perception of privacy changed in modern society or is it an expression of a perceived distance to the intangible digital media?